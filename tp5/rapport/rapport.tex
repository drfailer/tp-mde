\documentclass[a4paper]{article}

\usepackage[utf8]{inputenc}
\usepackage[T1]{fontenc}
\usepackage{textcomp}
\usepackage{multicol}
\usepackage{verbatim,listings,minted}
\usepackage{xcolor}
\usepackage{mathtools,amssymb,amsthm}
\usepackage[a4paper, total={6in, 8in}]{geometry}
\usepackage[francais]{babel}

%============header and foot============
\usepackage{fancyhdr}
\pagestyle{fancy}
\renewcommand\headrulewidth{1pt}
\fancyhead[L]{\bfseries TP5 MDE}
% \fancyhead[R]{\includegraphics[scale=0.05]{image.png}}
\fancyfoot[L]{CHASSAGNOL Rémi}
\fancyfoot[R]{2023-2024}

%================image=================
\usepackage{graphicx}

%==========rename listings=============
\renewcommand\listoflistingscaption{Table des extraits de codes}
\renewcommand\listingscaption{Extrait de code}

%------------------------------------------------------------------------------%
%                                   document                                   %
%------------------------------------------------------------------------------%

\begin{document}
\begin{titlepage}
\begin{center}

    \textsc{\LARGE TP IDM}\\[0.5cm]

    {\huge \bfseries Génération de flux parallèles de nombres pseudo-aléatoires\\[3cm]}

      \centering
      \includegraphics[scale=0.3]{./img/banner.png}

      \vspace*{\fill}

    \begin{multicols}{2}
      \large
      CHASSAGNOL Rémi\\

      \columnbreak

      \large
      \emph{Professeur:} HILL David\\
    \end{multicols}

    \textsc{\today}

  \end{center}
\end{titlepage}

\clearpage
\tableofcontents
\listoflistings
\clearpage

\section{Introduction}

L'objectif de ce TP est d'utiliser un générateur de nombres pseudo-aléatoires
pour réaliser des calculs en parallèle. Pour générer les nombres nous
utiliserons l'implémentation de Mersenne Twister fournie par la bibliothèque
CLHEP. Nous commencerons par détailler l'installation de la bibliothèque, puis
nous testerons la répétabilité des séquences de nombres générées aléatoirement
ainsi que le bon fonctionnement de la sauvegarde des statuts du générateur dans
des fichiers. Ensuite, nous traiterons l'exemple simple du calcul du nombre
$\pi$, d'abord en séquentiel, puis en parallèle. Enfin, nous traiterons un autre
exemple consistant à générer des séquences de bases nucléiques.\\

Pour pouvoir paralléliser les calculs nous commencerons par pré-générer des
fichiers de statuts de Mersenne Twister. En effet, ce générateur produit
toujours la même séquence de nombres (avec la seed fournie par Makoto
Matsumoto), or on ne souhaite pas repartir du même point dans chaque thread.
Pour que les threads utilisent des séquences différentes, nous utiliserons les
fichiers de statuts pour lancer le générateur à partir de point différents dans
la séquence. Pour ce faire, il faudra séparer les statuts d'un certain nombre
d'étapes (nombre à généré) lors de la génération. Nous discuterons des limites
de cette méthode dans la conclusion.

\clearpage

\section{Utilisation de CLHEP}

Dans cette première section nous allons voir comment installer la bibliothèque
CLHEP. Pour simplifier la gestion des dépendances, ce projet utilise des
sous-modules git.\\

Le clonage des sous-modules ne se fait pas automatiquement lors du clonage du
projet. Pour avoir accès aux sous-modules, il faut utiliser les commandes
de l'extrait de code \ref{git-sm}.

\begin{listing}[ht!]
\begin{minted}[frame=lines,framesep=2mm,baselinestretch=1.2,fontsize=\normalsize,linenos]{bash}
  git submodule init
  git submodule update
\end{minted}
\caption{Clonage des sous-modules git}
\label{git-sm}
\end{listing}

L'utilisation des sous-modules évite surcharger les serveurs contenant les
dépôts en stockant plusieurs fois le même code. De plus, ils permettent de
facilement cloner les dépendances (cela est plus simple que de télécharger la
bibliothèque manuellement).\\

Pour compiler la bibliothèque, il faut utiliser le script \texttt{build.sh}
du répertoire \texttt{lib}. Ce script va permettre de compiler et d'installer
la bibliothèque (fichiers d'entêtes, bibliothèque statiques, \dots) dans le
répertoire \texttt{lib/CLHEP-lib}. Ensuite, on peut compiler le projet en
utilisant CMake (utiliser un répertoire \texttt{build/} à la racine du
projet). Malheureusement, le CMake de la bibliothèque ne fournit pas de moyen
simple de compiler uniquement la partie \textit{Random} qui est utilisée pour ce
tp. La compilation peut donc être assez longue car le script compile tous les
modules de la bibliothèque.

À noter que pour compiler la bibliothèque CLHEP, le script \texttt{build.sh}
utilise \texttt{make} avec l'option \texttt{-j} pour paralléliser le
compilation. Cette option est pratique car elle permet de compiler les fichiers
plus rapidement, cependant, elle peut être consommatrice en ressources. En
effet, ici, on ne spécifie pas le nombre de threads, \texttt{make} utilise
toutes les ressources disponibles. Sur une machine peu puissante, il est
recommandé de modifier le script soit en spécifiant une limite après l'option,
soit la supprimant.\\

Maintenant que nous avons traité l'installation de la bibliothèque, nous allons
pourvoir commencer le tp.

\section{Test du générateur}

Avant de commencer les calculs, assurons-nous que le générateur produit bien
toujours les mêmes séquences de nombres lorsqu'il charge les fichiers de
statuts.\\

La générations des fichier de statuts se fait avec la méthode
\texttt{saveStatus} et le chargement des fichiers se fait avec
\texttt{restoreStatus}. Dans la fonction \texttt{checkReproducibility}, nous
générons dans un premier temps des fichiers de statuts, et chaque statuts est
séparé par un certain nombre de tirages de nombres pseudo-aléatoires. Les
tirages sont sauvegardés dans un tableau. Dans un second temps, nous restaurons
les statuts et nous vérifions que les tirages sont les mêmes que ceux
sauvegardés précédemment. Ici, nous utilisons un \texttt{assert} pour tester
l'égalité des nombres. Si les nouveaux nombres générés ne sont pas égaux à ceux
sauvegardés, le programme s'arrête avec un code erreur. Si le programme se
termine sans échouer, le test du générateur est valide.\\

À noter que ce test basique ne permet pas de valider avec certitude le bon
fonctionnement du générateur. Nous nous contenterons cependant de ce test étant
donné le fait la répétabilité de Mercenne Twister peut être prouvée
mathématiquement. Ici, nous vérifions juste que l'implémentation fonctionne sur
un petit nombre de tirages.

\section{Calcul de pi}

Maintenant que nous avons vérifié le bon fonctionnement du générateur, nous
allons réaliser un simple calcul de $\pi$ en utilisant la méthode de Monte
Carlo. Le calcul sera fait de façon séquentielle dans un premier temps, puis de
façon parallèle dans un second.

\subsection{Génération des fichiers de statuts}

Pour pouvoir paralléliser les calculs, il faut préalablement générer des
fichiers de statuts pour pouvoir lancer le générateur à partir d'un point donné.
Sans cette étape, les calculs lancés en parallèles utiliseront la même séquence
de nombre pseudo-aléatoires, ce qui rendrait la parallélisation inutile.\\

La génération des fichiers de statuts se fait à l'aide de la fonction
\texttt{generateStatusFiles}. On peut configurer le nombre de fichiers à
générer ainsi que le nombre de tirages séparant chaque statuts. Pour la suite,
on utilisera les valeurs par défauts des paramètres. Il est important d'utiliser
cette fonction avant de tester les autres fonctionnalité du programme car les
fichiers de statuts seront utilisés dans les autres questions. Les fichiers
générés sont nommés mt\_1, mt\_2, \dots

\subsection{Calcul séquentiel}

Dans un premier temps, nous allons réaliser le calcul de $\pi$ de façon
séquentielle. Cela va permettre de vérifier notre algorithme et simplifiera la
parallélisation.\\

Pour le calcul séquentiel, nous utilisons la fonction \texttt{piSequential} qui
réalise plusieurs calculs de $\pi$ et qui calcule la moyenne des résultats pour
avoir un valeur approchée. Cette fonction prend en paramètre le nombre de
réplications du calcul ainsi que le nombre de tirages pour chaque réplication.
Pour chaque réplication, on restaure un statut du générateur puis on utilise la
fonction \texttt{computePi}. Ici, il n'est pas vraiment nécessaire de
restaurer les statuts du générateur à chaque itération, cependant, cela permet
de facilement tester le code qui sera parallélisé par la suite. Le code de la
fonction est visible sur l'extrait de code \ref{fn-qf}.
\clearpage

\begin{listing}[ht!]
\begin{minted}[frame=lines,framesep=2mm,baselinestretch=1.2,fontsize=\normalsize,linenos]{CPP}
void piSequential(size_t nbReplications, size_t nbDraws,
               const std::string &fileName) {
    CLHEP::MTwistEngine mt;
    double sumPI = 0;

    timerStart();
    for (size_t i = 0; i < nbReplications; ++i) {
        mt.restoreStatus(cat(fileName, i).c_str());
        sumPI += computePi(mt, nbDraws);
    }
    timerEnd();

    std::cout << std::setprecision(10) << "pi: " << sumPI / nbReplications
              << "; time: " << timerCountMs() << std::endl;
}
\end{minted}
\caption{Fonction piSequential}
\label{fn-qf}
\end{listing}

La fonction permet aussi de mesurer le temps de calcul en utilisant les macros
visibles sur le l'extrait de code \ref{macro-timer}. Ces macros permettent
d'avoir une approximation du nombre de mili-secondes écoulées durant un calcul.
Ici, on choisit une précision à la micro-seconde ce qui est très élevé étant
donné le fait que l'exécution des calculs peut prendre plusieurs secondes.
Cependant, cela reste intéressant s'il on souhaite faire des mesures plus
précises.

\begin{listing}[ht!]
\begin{minted}[frame=lines,framesep=2mm,baselinestretch=1.2,fontsize=\normalsize,linenos]{CPP}
#define timerStart() auto _start = std::chrono::high_resolution_clock::now();
#define timerEnd() auto _end = std::chrono::high_resolution_clock::now();
#define timerCountMs()                                                         \
    std::chrono::duration_cast<std::chrono::microseconds>(_end - _start)       \
            .count() /                                                         \
        1000.0
\end{minted}
\caption{Macro timer}
\label{macro-timer}
\end{listing}

Ici, le calcul prend plus d'une minute pour s'exécuter. Dans la partie suivante,
nous allons comparer ce temps de calcul avec celui obtenu avec la
parallélisation.

\subsection{Calcul parallèle}

Dans cette section nous allons paralléliser le calcul de $\pi$ avec la méthode
de Monte Carlo. Nous commencerons par utiliser un script bash qui crée des
processus, puis nous utiliserons les threads de la bibliothèque standard de C++.

\subsubsection{Utilisation d'un script}
\label{sec:script}

Ici, nous allons utiliser un script bash pour paralléliser les calculs.
L'objectif est de faire en sorte de pouvoir passer en argument du programme le
nom d'un fichier de statut à charger de sorte à pouvoir lancer plusieurs fois le
même programme en tâche de fond avec un fichier différent à chaque foi.\\

La fonction \texttt{piFromStatusFile} est une version simplifiée de la fonction
\texttt{piSequential} qui ne lance qu'une seule fois le calcul. La fonction
prend en argument le nom du fichier de statut à charger pour initialiser le
générateur.

Le script utilisé pour lancer les processus est visible sur l'extrait de code
\ref{script}. Ici, on commence par vérifier que le projet est bien compilé et
que les fichiers de statuts ont bien été générés (dans le cas contraire, on fait
le nécessaire) avec la fonction \texttt{checkSetup}. Ensuite on définit le nom
des fichiers de sortie du programme (le nom est configurable avec les arguments
du script), puis on lance 10 processus en tâche de fond (utilisation du
\texttt{\&} en fin de ligne) avec une boucle \texttt{for}. Une fois l'exécution
terminée, les résultats sont stockés dans les fichiers du répertoire
\texttt{out/}.

\begin{listing}[ht!]
\begin{minted}[frame=lines,framesep=2mm,baselinestretch=1.2,fontsize=\normalsize,linenos]{bash}
#!/usr/bin/env bash

set -euo pipefail

checkSetup() { }
checkSetup

# configure output file
outputFile=pi
if [ $# -eq 1 ]; then outputFile=$1; fi; mkdir -p out
# compute pi
for i in {0..9}; do
    ./build/tp5 "./build/mt_$i" > "./out/$outputFile$i.out" &
done
time wait; echo "done"
\end{minted}
\caption{Script compute\_pi.sh}
\label{script}
\end{listing}

À noter qu'à la fin du script on utilise la commande \texttt{wait} qui va
permettre d'attendre que tous les processus lancés en tâche de fond se terminent
pour quitter le programme. On emploie aussi la commande \texttt{time} pour
mesurer le temps de calcul total (on considère que le temps de lancement des
processus est négligeable, on ne le prend donc pas en compte ici). Le calcul
prend environ dix secondes ce qui est bien plus rapide que le temps de calcul
séquentiel.\\

Dans cette section, nous avons utilisé un script pour paralléliser les calculs.
Dans la section suivante, nous utiliserons les threads de la bibliothèque
standard C++.

\subsubsection{Utilisation des threads}

La seconde solution utilise les threads de la bibliothèque standard de C++. Ici,
on utilise la fonction \texttt{piParallel} disponible en deux versions, une
utilisant directement les threads, et une autre utilisant l'abstraction
\texttt{std::future}. La version utilisant \texttt{std::future} est visible sur
l'extrait de code \ref{fn-g6}. Les deux version utilisent la fonction
\texttt{piFromStatusFile} présentée dans la section \ref{sec:script}.

\begin{listing}[ht!]
\begin{minted}[frame=lines,framesep=2mm,baselinestretch=1.2,fontsize=\normalsize,linenos]{CPP}
void piParallel(const std::string &fileName, size_t nbDraw) {
    CLHEP::MTwistEngine mt;
    size_t i;

    timerStart();
    /* threads scope */ {
        std::future<void> pis[NB_REPLICATION];
        for (i = 0; i < NB_REPLICATION; ++i) {
            pis[i] = std::async(std::launch::async, piFromStatusFile, cat(fileName, i),
                                nbDraw);
        }
    }
    timerEnd();
    std::cout << "total time: " << timerCountMs() / 1000.0 << "s" << std::endl;
}
\end{minted}
\caption{Fonction piParallel}
\label{fn-q6}
\end{listing}

Pour cette solution, le temps de calcul est similaire à celui du script ce qui
est logique étant donné le fait que le même nombre de threads a été utilisé.

\clearpage
\section{Gattaca}

Dans cette section, nous allons traiter un nouvel exemple de calcul parallèle.
Ici, on va simuler la génération de séquences de bases nucléiques. Ces bases
sont symbolisées par des lettres avec quatre possibilités A, C, G et T. Dans un
premier temps, on va chercher à générer une séquence simple, "GATTACA". Ensuite,
on discutera de la génération de séquences plus complexes.\\

Pour générer une séquence avec un certain nombre de bases, on utilise la
fonction \texttt{generateSequence}. Cette fonction var permettre de générer
une séquence, sous forme de chaine de caractères, avec une taille configurable.

Ensuite, on utilise la fonction \texttt{generateNSequences} qui va permettre
de générer un certain nombre de séquences. Cette fonction prend en paramètre une
séquence à chercher. La fonction s'arrête lorsque cette séquence est trouvée (ou
lorsque le compteur atteint le nombre maximum d'itérations). Cette fonction est
faite pour être parallélisée.

Enfin, la fonction \texttt{gattaca}, va lacer la fonction
\texttt{generateNSequences} dans un certains nombre de threads et elle va
collecter les résultats. À la fin, la fonction affiche le nombre de tirages
moyen nécessaires pour trouver la séquence recherchée.\\

À noter qu'avant de lancer la simulation, il faut générer les fichiers de
statuts du générateur. Pour ce faire, il faut utiliser les options
\texttt{gattaca} et \texttt{gen} comme sur l'extrait de code \ref{cmds-gattaca}.

\begin{listing}
\begin{minted}[frame=lines,framesep=2mm,baselinestretch=1.2,fontsize=\normalsize,linenos]{bash}
  ./tp5 gattaca gen # génération des fichiers de statuts
  ./tp5 gattaca     # lancement de la simulation
\end{minted}
\caption{Lancement de la simulation GATTACA}
\label{cmds-gattaca}
\end{listing}

Avec les règles énoncées précédemment, il y en tout $4^7 = 16,384$ possibilités
lorsque l'on génère des séquences de 7 bases (GATTACA est une séquence de 7
bases). En moyenne, il faut générer $11,351$ séquences pour trouver GATTACA (avec
le programme actuel), sachant que le nombre minimum d'essais trouvé par la
simulation est de $94$ et le nombre maximum est de $48,787$. L'écart type est
d'environ $11,789$.\\

La séquence GATTACA très courte et nombre de possibilités est assez faible. La
génération d'une séquence plus grande peut prendre beaucoup de temps. Par
exemple, pour chercher une séquence de 18 bases comme
\texttt{AAATTTGCGTTCGATTAG}, il y a en tout $4^{18} = 68,719,476,736$
possibilités. Ici, la plus grosse limitation vient de la génération des fichiers
de statuts. En effet, la génération des fichiers se fait de manière
séquentielle. Ici, si l'on souhaitait paralléliser la recherche sur $N$ threads,
il faudrait générer $N$ fichiers de statuts avec environ 70 milliards de tirage,
ce qui prendrait énormément de temps. Le problème de la génération de phrases
est encore plus complexe car ici, il y a encore plus de possibilités.

\clearpage
\section{Conclusion}

En conclusion, on peut dire que les problèmes de simulation stochastiques sont
facilement parallélisables. Cependant, avec des générateurs tels que Mersenne
Twister, on reste limité par le fait que l'on ne peut pas facilement lancer le
générateur à partir d'un certain point. La méthode employée ici consistait en la
génération de fichiers de statuts, cependant pour créer ces derniers il faut
tout de même faire des tirages de façon séquentielle avec le générateur ce qui
peut prendre un temps conséquent.\\

Ce problème pousse les créateur de générateurs à trouver de nouvelles méthodes.
Par exemple, Makoto Matsumoto a créé un \textit{"générateur de générateur"} qui
crée différents Mercenne Twister, et ce, de manière efficace. Malheureusement,
la fiabilité de ce générateur n'a pas encore été prouvée. Avec ce type de
générateurs, plus besoin de générer des fichiers de statuts, on peut directement
paralléliser les calculs.

\end{document}
